% Options for packages loaded elsewhere
\PassOptionsToPackage{unicode}{hyperref}
\PassOptionsToPackage{hyphens}{url}
\documentclass[
]{article}
\usepackage{xcolor}
\usepackage[margin=1in]{geometry}
\usepackage{amsmath,amssymb}
\setcounter{secnumdepth}{-\maxdimen} % remove section numbering
\usepackage{iftex}
\ifPDFTeX
  \usepackage[T1]{fontenc}
  \usepackage[utf8]{inputenc}
  \usepackage{textcomp} % provide euro and other symbols
\else % if luatex or xetex
  \usepackage{unicode-math} % this also loads fontspec
  \defaultfontfeatures{Scale=MatchLowercase}
  \defaultfontfeatures[\rmfamily]{Ligatures=TeX,Scale=1}
\fi
\usepackage{lmodern}
\ifPDFTeX\else
  % xetex/luatex font selection
\fi
% Use upquote if available, for straight quotes in verbatim environments
\IfFileExists{upquote.sty}{\usepackage{upquote}}{}
\IfFileExists{microtype.sty}{% use microtype if available
  \usepackage[]{microtype}
  \UseMicrotypeSet[protrusion]{basicmath} % disable protrusion for tt fonts
}{}
\makeatletter
\@ifundefined{KOMAClassName}{% if non-KOMA class
  \IfFileExists{parskip.sty}{%
    \usepackage{parskip}
  }{% else
    \setlength{\parindent}{0pt}
    \setlength{\parskip}{6pt plus 2pt minus 1pt}}
}{% if KOMA class
  \KOMAoptions{parskip=half}}
\makeatother
\usepackage{graphicx}
\makeatletter
\newsavebox\pandoc@box
\newcommand*\pandocbounded[1]{% scales image to fit in text height/width
  \sbox\pandoc@box{#1}%
  \Gscale@div\@tempa{\textheight}{\dimexpr\ht\pandoc@box+\dp\pandoc@box\relax}%
  \Gscale@div\@tempb{\linewidth}{\wd\pandoc@box}%
  \ifdim\@tempb\p@<\@tempa\p@\let\@tempa\@tempb\fi% select the smaller of both
  \ifdim\@tempa\p@<\p@\scalebox{\@tempa}{\usebox\pandoc@box}%
  \else\usebox{\pandoc@box}%
  \fi%
}
% Set default figure placement to htbp
\def\fps@figure{htbp}
\makeatother
\setlength{\emergencystretch}{3em} % prevent overfull lines
\providecommand{\tightlist}{%
  \setlength{\itemsep}{0pt}\setlength{\parskip}{0pt}}
\usepackage{bookmark}
\IfFileExists{xurl.sty}{\usepackage{xurl}}{} % add URL line breaks if available
\urlstyle{same}
\hypersetup{
  hidelinks,
  pdfcreator={LaTeX via pandoc}}

\author{}
\date{\vspace{-2.5em}}

\begin{document}

\section{IO\_WILI\_CGE}\label{io_wili_cge}

OBJETIVO:

Soft link entre el CGE y el IAM.

Para eso:

1.- Tienen que tener la misma desagregación de sectores ambas matrices,
para eso se ha hecho la matriz con los pesos por sectores agregados:
Coeficientes WILIAM Como este paso es vital, se ha generado un script de
comprobación: Comprobaciones\_WILI\_CGE

2.- Generar la tabla con las Z, para lo que se tiene que multiplicar la
tabla de coeficientes WILIAM por la matriz A de UNIZAR:
Matriz\_A\_PANTHEON Como este es vital, se ha generado un script de
comprobación: Comprobaciones\_Matriz\_A\_PANTHEON

3.- En el modelo de WILIAM la IO entra en tres diferentes scripts, para
los cuales, se necesita la IO (por lo que se multiplicará, la matriz por
las Z): IO\_PANTHEON Los ficheros que convierten la nueva IO en las
import shares intermidiate y IS por origin las tecnical coefficient
total,m se encuentran en otro documento. UNIFICAR.

\begin{itemize}
\tightlist
\item
  Explicación de los diferentes ficheros y scripts
\end{itemize}

Preparación del entorno y carga de datos

\begin{verbatim}
Instala (solo la primera vez) y carga las librerías writexl y openxlsx, necesarias para leer y escribir archivos de Excel en R.
\end{verbatim}

Define la ruta y hoja de cálculo que contienen la matriz original
W.xlsx, la lee como data frame con nombres de fila, y la convierte a
matriz numérica para facilitar los cálculos posteriores. 2. Limpieza de
etiquetas y definición de grupos

\begin{verbatim}
Estandariza los nombres de filas y columnas eliminando espacios alrededor de guiones y extrayendo los identificadores numéricos que van al final de cada etiqueta (por ejemplo, “Sector-21” → id 21).
\end{verbatim}

Crea cinco subgrupos B1--B5 para ciertos sectores subdivididos,
destacando que B5 agrupa conjuntamente a los sectores con id 6 y 21. A
partir de esos subgrupos deriva dos conjuntos de índices:

\begin{verbatim}
A, los sectores no subdivididos;

C2 (equivalente a todos los B), los sectores subdivididos que requieren un tratamiento especial.
\end{verbatim}

Define un helper get\_grp\_indices que, dado un índice de fila o
columna, devuelve los índices que pertenecen al mismo subgrupo B; si no
pertenece a ninguno, indica que está en A. 3. Construcción de la matriz
de coeficientes W

\begin{verbatim}
Inicializa una matriz cuadrada W (mismas dimensiones y etiquetas que la original) llena de NA y recorre todas las combinaciones fila/columna para poblarla siguiendo reglas de normalización específicas según si la fila/columna pertenece a A o a alguno de los subgrupos B.

    A×A: fija los coeficientes a 1 (identidad).

    A×C2: normaliza cada entrada dividiéndola por la suma de la fila correspondiente restringida a las columnas subdivididas.

    B×A: normaliza por la suma de la columna dentro del subgrupo de la fila.

    B×B: normaliza por la suma total del bloque definido por los subgrupos de fila y columna.

    En cualquier otra combinación asigna NA.
\end{verbatim}

\begin{enumerate}
\def\labelenumi{\arabic{enumi}.}
\setcounter{enumi}{3}
\item
  Chequeos de calidad

  Ejecuta comprobaciones dinámicas para verificar que las
  normalizaciones cumplen con las reglas esperadas: identidad en A×A,
  sumas unitarias por fila (A×C2), por columna dentro de cada subgrupo
  (B×A) y por bloque entre subgrupos (B×B). Si detecta desviaciones
  mayores que la tolerancia 1e-8, informa los errores; en caso
  contrario, confirma que todos los chequeos pasaron.
\item
  Exportación

  Convierte la matriz W en data frame y la guarda en un nuevo archivo
  Coeficientes\_WILI.xlsx, produciendo la versión normalizada de los
  coeficientes lista para su uso posterior.
\end{enumerate}

\end{document}
